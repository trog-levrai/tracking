\documentclass{beamer}


\usepackage[frenchb]{babel}

\usepackage[T1]{fontenc}

\usepackage[utf8]{inputenc}


\usetheme{Warsaw}


\title{Le tracking sur le web}

\author{Ilan 'trog' Dubois}


\begin{document}
    \begin{frame}
        \titlepage
    \end{frame}
    \section{Fonctionnement du tracking}
    \subsection{Techniques conventionnelles}
    \begin{frame}{label=cookies}
        \frametitle{Les cookies}
        \begin{center}
            Mini text stocké dans votre navigateur ne pouvant être lu que par le site qui l'a déposé.
            En y laissant un identifiant un site peut facilement vous reconnaître lors de vos visites et reconstituer votre historique sur ce site.
            Les annonceurs utilisent cette pratique pour tracker à l'échelle de plusieurs sites, ce sont des tierces parties.
        \end{center}
    \end{frame}
    \begin{frame}{label=statistiques}
        \frametitle{Les statistiques}
        \begin{center}
            Pour connaître son audience, un site doit collecter des informations techniques diverses.
            Google et son service analytics est très largement utilisé (54\%) pour collecter ces données.
            Ce qui implique que toutes les visites effectuées sur ces sites sont aussi connues de Google.
        \end{center}
    \end{frame}
\end{document}

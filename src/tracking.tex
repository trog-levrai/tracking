\documentclass{beamer}


\usepackage[frenchb]{babel}

\usepackage[T1]{fontenc}

\usepackage[utf8]{inputenc}


\usetheme{Warsaw}

\title{Le tracking sur le web}

\author{Ilan 'trog' Dubois}

\AtBeginSection[]
{
    \begin{frame}
        \frametitle{Sommaire}
            \tableofcontents[currentsection]
    \end{frame}
}


\begin{document}
    \begin{frame}
        \titlepage
    \end{frame}
    \section{Fonctionnement du tracking}
    \subsection{Les raisons du tracking}
        \begin{frame}{label=publicite}
            \frametitle{La publicité}
            \begin{center}
                Les annonceurs publicitaires sont très intéressés par ces techniques.
                Elles permettent en effet de n'envoyer que des annonces qui ont de fortes chances de vous intéresser.
                Le profilage et donc la pour optimiser les dépenses et augmenter l'intérêt des potentiels clients.
            \end{center}
        \end{frame}
        \begin{frame}{label=surveillance}
            \frametitle{La surveillance}
            \begin{center}
                Ces données peuvent également utilisées pour effectuer une surveillance des internautes.
                En profilant un utilisateur, on peut prévoir son comportement (avec de très importantes marges d'erreur).
                Dans le cadre de la loi renseignement, les annonceurs peuvent potentiellement avoir à fournir votre historique aux ``boîtes noires``.
            \end{center}
        \end{frame}
    \subsection{Techniques conventionnelles}
        \begin{frame}{label=cookies}
            \frametitle{Les cookies}
            \begin{center}
                Mini text stocké dans votre navigateur ne pouvant être lu que par le site qui l'a déposé.
                En y laissant un identifiant un site peut facilement vous reconnaître lors de vos visites et reconstituer votre historique sur ce site.
                Les annonceurs utilisent cette pratique pour tracker à l'échelle de plusieurs sites, ce sont des tierces parties.
            \end{center}
        \end{frame}
        \begin{frame}{label=statistiques}
            \frametitle{Les statistiques}
            \begin{center}
                Pour connaître son audience, un site doit collecter des informations techniques diverses.
                Google et son service analytics est très largement utilisé (54\%) pour collecter ces données.
                Ce qui implique que toutes les visites effectuées sur ces sites sont aussi connues de Google.
            \end{center}
        \end{frame}
        \begin{frame}{label=boutons}
            \frametitle{Les boutons de partage}
            \begin{center}
                Les réseaux sociaux proposent souvent des boutons pour partager le contenu d'une page.
                Lors du chargement du bouton sur la page, celui-ci collecte également des statistiques sur votre visite.
                Les sites comme Facebook et Twitter peuvent ainsi reconstituer une part très important de votre historique.
            \end{center}
        \end{frame}
\end{document}
